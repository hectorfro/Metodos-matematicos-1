
\chapter*{Introducci\'on}
\setcounter{page}{1}
\pagenumbering{arabic}
Como toda obra, el contenido de este libro tuvo como motivaci�n inicial la insatisfacci�n con lo que estaba disponible y esa necesidad de discutir un conjunto de conceptos con el matiz personal de los autores. A lo largo de casi 10 a�os se fue dibujando esa ruta, donde confluyen: una presentaci�n abstracta, una herramienta de c�lculo algebraico y variados ejemplos de aplicaci�n proveniente de la m�s diversas �reas.  Hemos tratado de mostrar que los conceptos abstractos son �tiles porque engloban, bajo un mismo enfoque, una multiplicidad de aplicaciones que normalmente las percibimos aisladas.

Existe, desde hace mucho tiempo,  cierta resistencia en los Departamentos de F�sica en incluir en sus programas de docencia cursos para la ense�anza de herramientas de computaci�n cient�fica y c�lculo num�rico, tal vez debido a los altos costos de la mayor�a de estos programas, casi todos comerciales. Pensamos que la utilizaci�n  de herramientas computacionales enriquece enormemente el aprendizaje de los estudiantes ya que los ense�an a abordar los problemas desde diferentes puntos de vista, les ayuda a familiarizarse con determinados lenguajes de programaci�n y los incentiva a desarrollar sus propias t�cnicas de c�lculo.

Hemos decidido utilizar un sistema de computaci�n algebraico de domino p�blico (o software libre) como {\bf Maxima}  porque estos programas  tienen la capacidad de ofrecer al estudiante  toda una gama de herramientas de c�lculo simb�lico, num�rico y de visualizaci�n de muy alto rendimiento. {\bf Maxima} es un programa que no requiere conocimientos previos de lenguajes de programaci�n pero permitir� que el estudiante se familiarice con la sintaxis de programaci�n y la transici�n a los lenguajes del tipo Fortran, C, o Python ser� mas sencilla. 

Los contenidos que aqu� presentamos han sido utilizados en cursos de M�todos Matem�ticos para estudiantes de pregrado en F�sica de la Universidad de los Andes (M�rida-Venezuela) y m�s recientemente  en los cursos de pregrado y posgrado para estudiantes en F�sica e Ingenier�as en la Universidad Industrial de Santander (Bucaramanga-Colombia). 




